\begin{frame}
  \frametitle{Solving for $\psi_{g,r}(s_{k},t)$}
    \begin{itemize}
        \item The MOC/TRRM solution begins with the {\it characteristic
            equation}, a partial solution to Equation \ref{eq:nte} for
            $\psi_{g,r}(s_{k},t)$. This
            expression contains integratls of $Q_{g,r}(s_{k},t)$ and
            $\ppt \psi_{g,r}(s_{k},t)$
            with respect to $s_{k}$.
        \item To obtain a closed form expression for $\psi_{g,r}(s_{k},t)$, we
            must make evaluate the integrals with an appropriate approxmation
            for $Q_{g,r}(s_{k},t)$ and $\ppt \psi_{g,r}(s_{k},t)$
        \item The {\bf flat source approximation} is a common and mathematically
            simple choice to resolve the $Q_{g,r}(s_{k},t)$ integral
    \end{itemize}
\end{frame}

\begin{frame}
  \frametitle{Scalar Flux Solution}
    With the flat source approximation, we can get a closed form solution for
    the scalar flux in each region by applying some clever algebra to Equation
    \ref{eq:nte}: 
    \begin{equation}
        \label{eq:scalar-flux-cell-averaged-flat}
        \overline{\phi}_{g,r}(t) = \frac{4\pi}{\Sig(t)} \left[\Qbar(t) +
        \frac{1}{V_r}\sum_k \omega_k \Delta \psi_{g,r}(t) \right] -
        \frac{1}{\Sig(t)\overline{v_g}} \ddt \overline{\phi}_{g,r}(t)
    \end{equation}
\end{frame}

\begin{frame}
  \frametitle{Source Derivative Propogation}
  Source Derivative Propogation \cite{hoffman_td_2013} is a method to resovle
  the $\ppt \psi_{g,r}(s_{k},t)$ integral. It works by formulating an
  equation equivalent to the characteristic equation
  (Eq. \ref{eq:flat-characteristic-equation-delta}) for $\ppt
  \psi(s_{k},t)$. We propogate $\ppt \psi(s_{k},t)$ along
  characteristics like we do with $\psi(s_{k},t)$

  This time-derviative characteristic equation is obtained by
  \begin{enumerate}
      \item Taking the time-derivative of Equation \ref{eq:flat-characteristic-equation-delta}
      \item Making an approximation for the $\ppptt \psi_{s_{k},t}$ term.
      \item Using the resulting equation to solve for the integral term
  \end{enumerate}
\end{frame}


